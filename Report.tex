\documentclass[12pt, a4paper]{article}

% --- PACKAGES ---
\usepackage{amsmath} % For advanced math environments
\usepackage{graphicx} % To include images
\usepackage[margin=1in]{geometry} % To set margins
\usepackage{hyperref} % For clickable links (optional)
\usepackage{float} % For better figure placement with [H]

% --- DOCUMENT INFORMATION ---
\title{Foundations of Machine Learning (DA5400) \\ \large Assignment 1}
\author{Shreehari Anbazhagan \\ Roll No: DA25C020}
\date{\today} % Automatically uses the current date

% --- BEGIN DOCUMENT ---
\begin{document}

\maketitle % This command generates the title based on the info above

\section{Introduction}
This report presents the implementation and analysis of...

\section{Part 1: PCA and Kernel PCA}

    \subsection{Principal Component Analysis (PCA)}
        \subsubsection{Theoretical Background}
        The objective of PCA is to...
        
        \subsubsection{Implementation and Results (Q1.a)}
        The explained variance for each principal component is shown in Table 1...

    \subsection{Kernel PCA}
        \subsubsection{Theoretical Background}
        The kernel trick allows us to...
        
        \subsubsection{Results and Plots (Q1.b)}
        The following plots show the data projected onto the top two principal components...
        
        \subsubsection{Analysis and Best Kernel Selection (Q1.c)}
        Based on the visual results, the best kernel for this dataset is...

\section{Part 2: Clustering}

    \subsection{K-means Clustering}
        \subsubsection{Theoretical Background}
        K-means aims to minimize the within-cluster sum of squares (WCSS)...
        
        \subsubsection{Random Initializations (Q2.a)}
        Figure X shows the error function versus iterations for 5 different runs...
        
        \subsubsection{Voronoi Regions (Q2.b)}
        The Voronoi diagrams for K = \{2, 3, 4, 5\} are presented below...

    \subsection{Spectral Clustering}
        \subsubsection{Theoretical Background}
        Spectral clustering treats the data as a graph...
        
        \subsubsection{Kernel Selection and Results (Q2.c)}
        An RBF kernel was chosen because...
        
        \subsubsection{Alternative Eigenvector Mapping (Q2.d)}
        Using the argmax assignment rule resulted in...

\section{Conclusion}
In summary, this project demonstrated...


\end{document}